% Options for packages loaded elsewhere
\PassOptionsToPackage{unicode}{hyperref}
\PassOptionsToPackage{hyphens}{url}
%
\documentclass[
]{article}
\usepackage{amsmath,amssymb}
\usepackage{lmodern}
\usepackage{iftex}
\ifPDFTeX
  \usepackage[T1]{fontenc}
  \usepackage[utf8]{inputenc}
  \usepackage{textcomp} % provide euro and other symbols
\else % if luatex or xetex
  \usepackage{unicode-math}
  \defaultfontfeatures{Scale=MatchLowercase}
  \defaultfontfeatures[\rmfamily]{Ligatures=TeX,Scale=1}
\fi
% Use upquote if available, for straight quotes in verbatim environments
\IfFileExists{upquote.sty}{\usepackage{upquote}}{}
\IfFileExists{microtype.sty}{% use microtype if available
  \usepackage[]{microtype}
  \UseMicrotypeSet[protrusion]{basicmath} % disable protrusion for tt fonts
}{}
\makeatletter
\@ifundefined{KOMAClassName}{% if non-KOMA class
  \IfFileExists{parskip.sty}{%
    \usepackage{parskip}
  }{% else
    \setlength{\parindent}{0pt}
    \setlength{\parskip}{6pt plus 2pt minus 1pt}}
}{% if KOMA class
  \KOMAoptions{parskip=half}}
\makeatother
\usepackage{xcolor}
\usepackage[margin=1in]{geometry}
\usepackage{graphicx}
\makeatletter
\def\maxwidth{\ifdim\Gin@nat@width>\linewidth\linewidth\else\Gin@nat@width\fi}
\def\maxheight{\ifdim\Gin@nat@height>\textheight\textheight\else\Gin@nat@height\fi}
\makeatother
% Scale images if necessary, so that they will not overflow the page
% margins by default, and it is still possible to overwrite the defaults
% using explicit options in \includegraphics[width, height, ...]{}
\setkeys{Gin}{width=\maxwidth,height=\maxheight,keepaspectratio}
% Set default figure placement to htbp
\makeatletter
\def\fps@figure{htbp}
\makeatother
\setlength{\emergencystretch}{3em} % prevent overfull lines
\providecommand{\tightlist}{%
  \setlength{\itemsep}{0pt}\setlength{\parskip}{0pt}}
\setcounter{secnumdepth}{-\maxdimen} % remove section numbering
\usepackage{amsmath}
\usepackage{caption}
\ifLuaTeX
  \usepackage{selnolig}  % disable illegal ligatures
\fi
\IfFileExists{bookmark.sty}{\usepackage{bookmark}}{\usepackage{hyperref}}
\IfFileExists{xurl.sty}{\usepackage{xurl}}{} % add URL line breaks if available
\urlstyle{same} % disable monospaced font for URLs
\hypersetup{
  hidelinks,
  pdfcreator={LaTeX via pandoc}}

\title{\vspace{1in}Capstone Proposal\\
\vspace{.25in}No Man's Sky\\
Explored Space Map\\
\vspace{.5in}}
\usepackage{etoolbox}
\makeatletter
\providecommand{\subtitle}[1]{% add subtitle to \maketitle
  \apptocmd{\@title}{\par {\large #1 \par}}{}{}
}
\makeatother
\subtitle{CSCI 490\\
Computer Science\\
\&\\
Data Science\\
Capstone \vspace{.5in}}
\author{Derek Borders\\
\vspace{1in}}
\date{Last compiled on\\
September 27, 2022}

\begin{document}
\maketitle

\pagebreak

\hypertarget{no-mans-sky-interactive-3d-galaxy-map}{%
\section{No Man's Sky Interactive 3d Galaxy
Map}\label{no-mans-sky-interactive-3d-galaxy-map}}

No Man's Sky is a space exploration game that uses extensive procedural
generation to create a mind mindbogglingly massive open-world play
space. The game contains 255 galaxies, each of which contains 4.2
billion regions. Each region is a roughly 400 light-year cube of space
and home to around 100 to 600 star systems.\footnote{Each region has
  address space for 4096 systems but the vast majority of these are
  inaccessible `phantom' systems that do not appear on the in-game
  galaxy map and cannot be reached normally.} Finally, each system has
2-6 planets (counting moons).

The number of planets is so vast that the vast majority of them will
never be visited by anybody. Even with all users sharing a starting
galaxy (Euclid), in normal exploration, an individual player is often
the only person who has ever seen a given planet they land on.

Each world in No Man's Sky are addressed and reachable via an in-game
portal network using its unique portal address. This address uses a
unique 16 glyph system that translates nicely to hexadecimal. In each
address, the final 8 characters translate to hexadecimal coordinates in
3d space. Two for the y axis, and three each for the X and Z axis. These
coordinates identify the galactic region containing the system.

No Man's Sky includes a `photo mode' feature which includes an overlay
in the lower left corner displaying the current galactic coordinates in
glyphs. Using this feature with the screenshot feature, users can take
and share images of their travels with coordinates included. An entire
community of players engage in this activity on
reddit.com/r/NMSCoordinateExchange.

The goal of the proposed project is to create a tool that can take in
these screen captures, use machine learning to extract the glyphs,
translate them to coordinates, and plot the regions in an intuitive,
interactive 3D visualization (or several) in some kind of dashboard UI.
Mapping will be from both a local directory of personally captured
images representing my own explored space, as well as images scraped
from Reddit to stand in as an approximation of globally explored space.

The project would initially start as a local-only Python analysis tool
(package), with the possibility of being extended into a small web
application.

\hypertarget{reason-for-selection}{%
\subsection{Reason for Selection}\label{reason-for-selection}}

I came up with this idea as a way to get some exposure to basic neural
networks and computer vision while doing something I am personally
interested in. I also expect to be able to further develop my data
visualization skills using some tools I've been aware of but have not
had the opportunity to explore.

\hypertarget{data-science}{%
\subsection{Data Science}\label{data-science}}

I think the image classification, machine learning, and data
visualization elements here are enough to qualify this as a data science
project. That said, I will be looking for ways to perform additional
analysis on the information I scrape from the web. This project is much
more descriptive than predictive analytics. While I understand the DS
certificate program's focus on predictive analytics, I think descriptive
analytics, particularly for spatial things like mapping, are very
interesting and worth exploring further.

\pagebreak

\hypertarget{features}{%
\subsection{Features}\label{features}}

\hypertarget{core-features}{%
\subsubsection{Core Features}\label{core-features}}

\begin{itemize}
\tightlist
\item
  mapping personally explored space against space explored the Reddit
  community as a whole
\item
  options to switch between several different visualizations:

  \begin{itemize}
  \tightlist
  \item
    3d scatter plot
  \item
    3d surface plot\\
  \item
    2d heat map
  \end{itemize}
\item
  various quantitative analyses of community explored space and how
  users, planets, systsems, and regions all relate to one another
\end{itemize}

\hypertarget{additional-features-aspirational}{%
\subsubsection{Additional Features
(aspirational)}\label{additional-features-aspirational}}

\begin{itemize}
\tightlist
\item
  A more advanced computer vision model that can additional determine
  the subject of a given image:

  \begin{itemize}
  \tightlist
  \item
    umbrella categorization:

    \begin{itemize}
    \tightlist
    \item
      a planet in general\\
    \item
      a star ship, freighter, or frigate\\
    \item
      fauna\\
    \item
      multitool\\
    \end{itemize}
  \item
    subcategories

    \begin{itemize}
    \tightlist
    \item
      lush planet\\
    \item
      `strider' or `diplo' fauna\\
    \item
      fan-wing hauler, x-wing fighter\\
    \item
      alien multitool\\
    \end{itemize}
  \end{itemize}
\item
  the ability to hover over a given region and see some summary
  statistics about it

  \begin{itemize}
  \tightlist
  \item
    number of explored systems\\
  \item
    known individuals to have explored the region\\
  \item
    closest explored region\\
  \end{itemize}
\item
  the ability to hover over a given region and the image or gallery of
  images related to it\\
\item
  a visualization of regions that appear to be unexplored\\
\item
  a way to generate possible coordinates for the system farthest from
  explored space\\
\item
  support for galaxies beyond the first\\
\item
  a publicly accessible web application (low priority)
\end{itemize}

\hypertarget{tools}{%
\subsection{Tools}\label{tools}}

\begin{itemize}
\tightlist
\item
  Python

  \begin{itemize}
  \tightlist
  \item
    Pandas, Numpy, PyTorch, Plotly Dash\\
  \item
    PRAW (Python Reddid API Wrapper)
  \end{itemize}
\item
  TensorFlow, OpenCV

  \begin{itemize}
  \tightlist
  \item
    possibly other, more appropriate ML/CV tools\\
  \item
    possibility of adapting an existing OCR engine (like Tesseract)\\
  \item
    possibly a transformer based on an imagenet model (to categorize
    images)\\
  \end{itemize}
\item
  Nvidia RAPIDS suite for GPU acceleration\\
\item
  Docker\\
\item
  HTML, JS, CSS\\
\item
  Default to Django, possibly React or some other framework

  \begin{itemize}
  \tightlist
  \item
    If I get to a web interface, I'd rather not build it from scratch
  \end{itemize}
\end{itemize}

\hypertarget{scope-tiers}{%
\subsection{Scope Tiers}\label{scope-tiers}}

\hypertarget{stage-1.-proof-of-concept---personal-map}{%
\subsubsection{Stage 1. Proof of Concept - Personal
Map}\label{stage-1.-proof-of-concept---personal-map}}

\begin{itemize}
\tightlist
\item
  Scan local directory of personal images

  \begin{itemize}
  \tightlist
  \item
    Identical format\\
  \item
    Single galaxy, platform, \& version\\
  \end{itemize}
\item
  Process images\\
\item
  Extract coordinates\\
\item
  Visualize with interactive plot

  \begin{itemize}
  \tightlist
  \item
    base plotly
  \end{itemize}
\end{itemize}

\hypertarget{stage-2.-dashboard-ui}{%
\subsubsection{Stage 2. Dashboard UI}\label{stage-2.-dashboard-ui}}

\begin{itemize}
\tightlist
\item
  Upgrade to modular dahsboard UI

  \begin{itemize}
  \tightlist
  \item
    Plotly Dash\\
  \end{itemize}
\item
  Modular design for efficient addition of later features\\
\item
  Plotly Dash (or other framework that easily extends to web
  applications)
\end{itemize}

\hypertarget{stage-2.-feature-collective-map}{%
\subsubsection{Stage 2. Feature: Collective
Map}\label{stage-2.-feature-collective-map}}

\begin{itemize}
\tightlist
\item
  Scrape reddit for a collection of images\\
\item
  Start with only initial galaxy, pc platform, \& update 3.94\\
\item
  Include overlaid map and separate personal/collective map
\end{itemize}

\hypertarget{stage-3.-scale-increase}{%
\subsubsection{Stage 3. Scale Increase}\label{stage-3.-scale-increase}}

\begin{itemize}
\tightlist
\item
  Add images from older updates, other platforms, additional galaxies,
  etc
\end{itemize}

\hypertarget{stage-4.-upgrade-with-miscellanous-optional-features}{%
\subsubsection{Stage 4. Upgrade with Miscellanous Optional
Features}\label{stage-4.-upgrade-with-miscellanous-optional-features}}

\begin{itemize}
\tightlist
\item
  Identify most isolated regions and generate random coordinates to
  attempt jumping to unexplored space.
\item
  Find an efficient way to make it so you can hover over points and see
  the image(s) related to that point.\\
\item
  Scrape with flair searches to add filterable attributes like galaxy,
  subject, platform\\
\item
  Use subject specific scraped images to train a transformer to
  differentiate between base, ship, multi-tool, freighter, and
  flora/fauna\\
\item
  Process uploaded images to `paint out' undesirable obstructions
  (UI/HUD elements) with ML-based technique\\
\item
  Galactic population density heat map

  \begin{itemize}
  \tightlist
  \item
    Individual regions have hundreds of systems\\
  \item
    With sufficient sample size, collisions are likely\\
  \item
    Some regions or clusters of regions may have images from dozens or
    hundreds of systems/planets
  \end{itemize}
\end{itemize}

\hypertarget{stage-5.-publish-to-web-application}{%
\subsubsection{Stage 5. Publish to Web
Application}\label{stage-5.-publish-to-web-application}}

\begin{itemize}
\tightlist
\item
  Containerization w/ Docker\\
\item
  Basic web interface
\end{itemize}

\end{document}
